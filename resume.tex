\documentclass[%singlesided,
               doublesided,
               paper=a4,
               fontsize=11pt
              ]{my-resume}


%%%%%%%%%%%%%%%%%%%%%%%%%%%%%%%%%%%%%%%%%%%%%%%%%%%%%%%%%%%%%%%%%%%%%%%%%%%%%%%%
% set geometry
%%%%%%%%%%%%%%%%%%%%%%%%%%%%%%%%%%%%%%%%%%%%%%%%%%%%%%%%%%%%%%%%%%%%%%%%%%%%%%%%

\setlength\highlightwidth{7.5cm}
\setlength\headerheight{1cm}            % note that margintop gets added to this value, i.e. the header bar is 5cm
\setlength\marginleft{0.3cm}
\setlength\marginright{\marginleft}      % needs to be 1.5 times to be actually equal. why?
\setlength\margintop{1cm}
\setlength\marginbottom{1cm}


%%%%%%%%%%%%%%%%%%%%%%%%%%%%%%%%%%%%%%%%%%%%%%%%%%%%%%%%%%%%%%%%%%%%%%%%%%%%%%%%
% FONTS
%%%%%%%%%%%%%%%%%%%%%%%%%%%%%%%%%%%%%%%%%%%%%%%%%%%%%%%%%%%%%%%%%%%%%%%%%%%%%%%%

\RequirePackage{fontspec}
\setmainfont{Carlito}
% Carlito


%%%%%%%%%%%%%%%%%%%%%%%%%%%%%%%%%%%%%%%%%%%%%%%%%%%%%%%%%%%%%%%%%%%%%%%%%%%%%%%%
% COLORS
%%%%%%%%%%%%%%%%%%%%%%%%%%%%%%%%%%%%%%%%%%%%%%%%%%%%%%%%%%%%%%%%%%%%%%%%%%%%%%%%

\colorlet{highlightbarcolor}{white}
\colorlet{headerbarcolor}{white}

\colorlet{headerfontcolor}{black}
\colorlet{accent}{black}
\colorlet{heading}{black}
\colorlet{emphasis}{black}
\colorlet{body}{black}

\hypersetup{
    colorlinks=true,
    linkcolor=blue,
    filecolor=magenta,      
    urlcolor=cyan,
}
%%%%%%%%%%%%%%%%%%%%%%%%%%%%%%%%%%%%%%%%%%%%%%%%%%%%%%%%%%%%%%%%%%%%%%%%%%%%%%%%
% set document
%%%%%%%%%%%%%%%%%%%%%%%%%%%%%%%%%%%%%%%%%%%%%%%%%%%%%%%%%%%%%%%%%%%%%%%%%%%%%%%%


\begin{document}

\name{Yuan He}
\tagline{PhD Student in Computer Science}
% \photo[round]{picture.jpg}{\dimexpr \headerheight-\marginbottom}   % make photo exactly match the header with margintop/marginright/marginbottom as margin

\makeheader

\highlightbar{

    \section{Contact}
    
    \email{yuan.he@cs.ox.ac.uk}
    % \phone{+49 123 456789}
    \location{Parks Rd, Oxford, OX1 3QD}
    % \vspace{0.5em}
    \homepage{Personal Homepage}{https://lawhy.github.io/}
    
    \github{Github Profile}{https://github.com/Lawhy}
    
    \linkedin{LinkedIn Profile}{https://www.linkedin.com/in/yuan-he-0557781aa/}
    
    \googlescholar{Google Scholar Profile}{https://scholar.google.com/citations?view_op=list_works&hl=en&user=sCXUhQcAAAAJ}
    % \orcid{0000-0003-1104-2014}{https://www.orcid.org/0000-0003-1104-2014}
    % \ads{NASA/ADS publication list}{https://ui.adsabs.harvard.edu/search/fq=\%7B!type\%3Daqp\%20v\%3D\%24fq\_database\%7D&fq\_database=database\%3A\%20astronomy&p\_=0&q=pubdate\%3A\%5B2016-01\%20TO\%209999-12\%5D\%20author\%3A(\%22Krieger\%2C\%20Nico\%22)&sort=date\%20desc\%2C\%20bibcode\%20desc}
    
    \section{Skills}
    
    \skillsection{Expertise}
    \skill{Knowledge Graphs}{5}
    \skill{Machine Learning}{5}
    \skill{Natural Language Processing}{5}
    
    \divider
    
    \skillsection{Programming}
    \skill{Python, Pytorch}{5}
    \skill{Latex, Bash}{5}
    \skill{Java, Kotlin, Android}{4}
    \skill{Matlab, R, Maple}{4}
    \skill{Haskell, HTML/CSS}{4}
    \skill{C/C++, MIPS}{3}
    % \skill{Apple Script}{3}
    % \skill{HTML/CSS}{3}
    % \skill{LaTeX}{4}
    
    % \vspace{0.5em}
    % \skillsection{Operating Systems}
    % \skill{Linux}{3}
    % \skill{MacOS}{5}
    % \skill{Windows}{5}
    
    % \vspace{0.5em}
    % \skillsection{Software \& Tools}
    % \skill{Visualisation}{5}
    % (e.g. matplotlib, seaborn, ...)\\
    % \skill{Data handling/analysis}{5}
    % (e.g. numpy, scipy, pandas, ...)\\
    % \skill{Deep Learning Framework}{5}
    % (e.g. Pytorch)
    % \skill{Docker}{3}
    % \skill{Office}{4}
    
    % \vspace{0.5em}
    % \skillsection{Another skill subsection header}
    % You can also put simply text here without the dots.
    
    \divider
    
    \skillsection{Languages}
    \skill{Mandarin}{5}
    \skill{English}{5}
    
    \divider
    
    \section{Honors \& Awards}
    
    \achievement{Received the Joint Class Prize for Top Performance in AI \& Maths at School of Informatics, University of Edinburgh.}{10/2020}{\href{https://github.com/Lawhy/Lawhy.github.io/raw/master/materials/Certificate\%20-\%20Y.\%20He\%5B1591\%5D.pdf}{Certificate}}
    
    \divider
    
    \achievement{Received Web of Data Course Certificate issued by EIT Digital through Coursera.}{11/2020}{\href{https://coursera.org/share/08b65b727ca6caebb7d6edc3ab67f061}{Certificate}}
    
    \divider
    
    \achievement{Won the Golden Award in China College Students' 'Internet+' Innovation and Entrepreneurship Competition.}{11/2020}{\href{https://cy.ncss.cn/information/8a80808d758bf61b0175da8689820041}{Award List}}
    
    \divider
    
    \achievement{First Prize in KYOTO International Entrepreneurship Contest For University Students.}{12/2020}{\href{https://www.ie.education/}{Award List}}
    
    \divider
    

}
\mainbar{
    % \section{About this template}
    % Section are set in bold face. An optional parameter of \texttt{\textbackslash section} takes a symbol to add in front of the text. This option is used in the jobs and education sections below.
    
    \section[\faMortarBoard]{Education}
    
    \cvevent{University of Oxford} {Doctor of Philosophy in Computer Science } {10/2020 -- Present} {Oxford, UK} 
    \divider
    
    \cvevent{University of Edinburgh}{BSc (Hons) Artificial Intelligence and Mathematics}{09/2016 -- 07/2020}{Edinburgh, UK}
    \hbox {Ranked Top 1 in the degree of Artificial Intelligence and Mathematics.}
    \divider

    
    \section[\faGears]{Experiences}
    % \cvevent{Funded Doctoral Researcher}{}{Dec 2020}{Online}
    % \hbox{\small Present the paper (see Publications) accepted by AACL-IJCNLP virtually \href{https://www.youtube.com/watch?v=VzcCkX_MjWU&t=15s}{[video]}.}
    % \divider 
    
    
    \cvevent{Conference Presentation}{AACL-IJCNLP 2020 }{Dec 2020}{Online}
    \hbox{\small Present the paper (see Publications) accepted by AACL-IJCNLP virtually \href{https://www.youtube.com/watch?v=VzcCkX_MjWU&t=15s}{[video]}.}
    \divider 
    
    \cvevent{MAT Marker}{Mathematical Institute, University of Oxford}{Nov 2020}{Oxford, UK}
    \hbox{\small Participate in the marking of Maths Admissions Test (MAT), which is an important}
    \hbox{\small test for shortlisting candidates for the interview of entering University of Oxford.}
    \divider 
    
    \cvevent{Teaching Assistant}{School of Informatics, University of Edinburgh}{Jan 2019 -- May 2019}  {Edinburgh, UK}
    \hbox{\small Undertake the role of Lab Demonstrator in the course Reasoning \& Agents.}
    \divider
    
    \cvevent{Research Intern}{Edinburgh NLP Group, University of Edinburgh}{Jun 2018 -- Aug 2018}     {Edinburgh, UK}
    \hbox{\small Working for the NLP research project: Multilingual Machine Transliteration.}
    \divider
    
    \cvevent{Research Assistant}{Business School, University of Edinburgh}{May 2017 -- Dec 2019}     {Edinburgh, UK}
    \hbox{\small Working for several NLP \& Coding in Finance projects led by Dr Hang Zhou.}
    \divider
    
    \cvevent{Software Intern}{S\&S IT}{Jul 2017 -- Aug 2017}{Singapore}
    \hbox{\small Working for the development of an accounting software.}
    \divider
    
    % \job{date}
    %     {Employer name and city}
    %     {Position Title}
    %     {Additional details for this position. Can be left empty to omit this line}
    % \job{01/2020 - 12/2020}
    %     {An employer with a long name,\\City in new line}
    %     {Previous position}
    %     {}
    

    % \section{Achievements, honours and awards}
    % \achievement{My first achievement}
    % \achievement{My second achievement}

    % \section{General Skills}
    % \smallskip % additional skip because tag outlines use up space
    % \tag{Tag 1}
    % \tag{Tag 2}
    % \tag{and}
    % \tag{another tag}
    % \tag{some more tags}
    % \tag{yet another one}
    % \tag{tags flow over}
    % \tag{to the next line}
    % \tag{if necessary}
    
%     \publication
% 	{English-to-Chinese Transliteration with Phonetic Auxiliary Task} % Title
% 	{\textbf{Yuan He}, Shay B. Cohen \ss} % Authors
% 	{Dec 2020} % Year
% 	{The Astrophysical Journal Vol.897, Issue 2, id.176} % Journal
% 	{\ADS{https://ui.adsabs.harvard.edu/abs/2020ApJ...897..176K}, \arXiv{https://arxiv.org/abs/2006.08262}} % ADS & arxiv links
    
    % \section{Wheel Chart}
    % % This is taken from AltaCV
    % % see https://github.com/liantze/AltaCV for details
    % \wheelchart{1.5cm}{0.5cm}{% outer and inner diameter
    %     6/8em/accent!20/Sleep,          % comma-separated list of
    %     8/8em/accent!40/Daytime job,    % fraction of 24 / line length / color / label
    %     2/8em/accent!80/Training,          % here, the color is shades of the accent color
    %     3/8em/accent!60/Recovering from fighting criminals,
    %     5/8em/accent/Being Batman
    % }
}
\makebody
\clearpage
\section[\faBook]{Publications}
\nocite{*}
\printbibliography[heading=none]

\section[\faFolder]{Projects}

\textbf{Multilingual Machine Transliteration}
\smallskip
\begin{itemize}
\item Initially, it was a short-term research project aimed at exploring NLP techniques for improving the neural model on the multilingual transliteration task. It was later extended to my undergraduate final project and our work has led to a publication about English-to-Chinese transliteration (see Publications).
\end{itemize}
\divider

\textcolor{gray}{Two NLP \& Coding in Finance research projects led by Dr Hang Zhou.}
\smallskip

\textbf{Social Media, Financial Reporting Opacity and Return Co-movement: Evidence from Seeking Alpha}
\smallskip
\begin{itemize}
    \item Data collection task: extracting finance social media data using web crawler technique.
    \item The relevant paper is accepted at the Journal of Financial Markets.
\end{itemize}
\smallskip

\textbf{Private In-house Meeting and Crash Risk}
\begin{itemize}
    \item Textual analysis tasks: \ding{172} Analyse the readability \& sentiment of the firm specific information of companies disclosed by through investor in-house meeting reports; 
    \ding{173} Compute the similarities among investor in-house meeting reports;
    \ding{174} Build a general machine learning model for documents in bag-of-words representation to select useful information from the investor in-house meeting reports.
    \item The relevant paper is under revision and resubmission.
\end{itemize}
\divider

\textbf{Search Engine Development: Searchive}
\smallskip
\begin{itemize}
    \item A group project aimed to develop a prototype search engine used for searching the scholar articles stored in \href{https://arxiv.org/}{arxiv}. 
\end{itemize}
\divider

\textbf{Android Game Development: Coinz}
\smallskip
\begin{itemize}
    \item An individual project aimed at designing and implementing a multiplayer online game called Coinz.   The  basic  activities  of  this  game  are  collecting,  exchanging  virtual  coins  and  discovering  a strategy of becoming ’richer’ than other players.
\end{itemize}
\divider

\textbf{Business Plan for the project: Implantable Microsystems for Personalized Anti-cancer Therapy (\href{https://www.eng.ed.ac.uk/research/projects/impact-implantable-microsystems-personalised-anti-cancer-therapy}{IMPACT})}
\smallskip
\begin{itemize}
    \item Our business plan for the IMPACT system has won top prizes in some venture contests (see HONORS \& AWARDS).
    \item IMPACT is a 5-year, £5.2M research project, funded by an EPSRC Programme Grant, to develop new approaches to cancer treatment, using implanted, smart sensors on silicon, fabricated in the University's Scottish Microelectronics Centre.
\end{itemize}
\divider

% \pagestyle{highlightmain}
% \highlightbar{}
% \mainbar{

%     \section{Another section}
    
%     This page uses the page style \texttt{highlightmain} which shows the highlight bar (gray) and the main part (white background) but omits the header. 
%     The default page style is \texttt{headerhighlightmain} with all three elements.
%     If you don't want header, nor highlight bar, use page style \texttt{\textbackslash pagestyle\{empty\}}.
%     \medskip
%     Neither main, nor highlight bar must be filled to make this template work.
%     It is possible to use a page style with the highlight bar but leave it empty by setting an empty highlightbar \texttt{\textbackslash highlightbar\{\}}.

%     \vspace{0.5em}
%     \subsection{Subsection 1}
%     Demonstrate subsections.
    
%     \subsection{Subsection 2}
%     Subsection are also bold face but a smaller font then section. They also omit the rule.
    

% }
% \makebody


% \clearpage
% \pagestyle{empty}

% \section{Publications}
% \pubforcefullwidth

% Demonstrate what an \texttt{\textbackslash pagestyle\{empty\}} page looks like.
% Also show off the macros for publications that uses small icons for authors, date, journal and links.

% Achieving a good looking spacing can be tricky. For empty pagestyles where the full width is available use \texttt{\textbackslash pubforcefullwidth} to force the publoication list to take up all the available space.
% The (relative) lengths reserved for date, journal and links can be set with the parameters \texttt{\textbackslash pubdatelength}, \texttt{\textbackslash pubjournallength} and \texttt{\textbackslash publinklength} as in \texttt{\textbackslash setlength\{\textbackslash pubdatelength\}\{0.15 \textbackslash linewidth\}}.
% \bigskip

% \publication
% 	{The turbulent gas structure in the centers of NGC~253 and the Milky Way} % Title
% 	{\textbf{N. Krieger}, A. Bolatto, E. Koch, A. Leroy, E. Rosolowsky, F. Walter, A. Wei\ss, D. Eden, R. Levy, D. Meier, E. Mills, T. Moore, J. Ott, Y. Su, S. Veilleux} % Authors
% 	{2020} % Year
% 	{The Astrophysical Journal Vol. 899, Issue 2, id.158} % Journal
% 	{\ADS{https://ui.adsabs.harvard.edu/abs/2020ApJ...899..158K}, \arXiv{https://arxiv.org/abs/2008.02518}} % ADS & arxiv links

% \publication
% 	{The molecular ISM in the Super Star Clusters of the starburst NGC253} % Title
% 	{\textbf{N. Krieger}, A. Bolatto, A. Leroy, R. Levy, E. Mills, D. Meier, S. Veilleux, F. Walter, A. Wei\ss} % Authors
% 	{2020} % Year
% 	{The Astrophysical Journal Vol.897, Issue 2, id.176} % Journal
% 	{\ADS{https://ui.adsabs.harvard.edu/abs/2020ApJ...897..176K}, \arXiv{https://arxiv.org/abs/2006.08262}} % ADS & arxiv links

% \publication
% 	{The Molecular Outflow in NGC\,253 at a Resolution of Two Parsecs} % Title
% 	{\textbf{N. Krieger}, A. Bolatto, F. Walter, A. Leroy, L. Zschaechner, D. Meier, J. Ott, A. Wei\ss, E. Mills, S. Veilleux, M. Gorski} % Authors
% 	{2019} % Year
% 	{The Astrophysical Journal Vol.881, Issue 1, article id. 43, 20 pp} % Journal
% 	{\ADS{https://ui.adsabs.harvard.edu/abs/2019ApJ...881...43K}, \arXiv{https://arxiv.org/abs/1907.00731}} % ADS & arxiv links

\end{document}
